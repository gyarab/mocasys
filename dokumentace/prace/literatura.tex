%%% Seznam použité literatury (bibliografie)
%%%
%%% Pro vytváření bibliografie používáme bibTeX. Ten zpracovává
%%% citace v textu (např. makro \cite{...}) a vyhledává k nim literaturu
%%% v souboru literatura.bib.
%%%
%%% Příkaz \bibliographystyle určuje, jakým stylem budou citovány odkazy
%%% v textu. V závorce je název zvoleného souboru .bst. Styly plainnat
%%% a unsrt jsou standardní součástí latexových distribucí. Styl czplainnat
%%% je dodáván s touto šablonou a bibTeX ho hledá v aktuálním adresáři.

\bibliographystyle{czplainnat}    %% Autor (rok) s českými spojkami
% \bibliographystyle{plainnat}    %% Autor (rok) s anglickými spojkami
% \bibliographystyle{unsrt}       %% [číslo]

\renewcommand{\bibname}{Seznam použité literatury}

%%% Vytvoření seznamu literatury. Pozor, pokud jste necitovali ani jednu
%%% položku, seznam se automaticky vynechá.

\bibliography{literatura}

%%% Kdybyste chtěli bibliografii vytvářet ručně (bez bibTeXu), lze to udělat
%%% následovně. V takovém případě se řiďte normou ISO 690 a zvyklostmi v oboru.

% \begin{thebibliography}{99}
%
% \bibitem{lamport94}
%   {\sc Lamport,} Leslie.
%   \emph{\LaTeX: A Document Preparation System}.
%   2. vydání.
%   Massachusetts: Addison Wesley, 1994.
%   ISBN 0-201-52983-1.
%
% \end{thebibliography}
