\chapter*{Úvod}
\addcontentsline{toc}{chapter}{Úvod}

Tato dokumentace a popsaný systém, Mocasys, jsme vytvořili jako skupinovou
ročníkovou práci pro předmět Programování ve 3. ročníku. Volbu tématu ovlivnilo
několik faktorů. Jeden z nich byla komplexita práce, která měla být adekvátní
týmu čtyř kompetentních programátorů. Dalším faktorem byla využitelnost
výsledného programu. Námi vytvořený systém pro jídelny může být, po malých
úpravách pro jednotlivé provozovny, použit jako náhrada existujícího komerčního
software.

Práci jsme pojali částečně i jako experiment v oblasti nástrojů a knihoven pro
tvorbu webových aplikací. Vytvořili jsme databázový framework DASCore, který
umožňuje jednoduše vytvářet databáze s ukládáním historie a ověřováním
uživatelských oprávnění na úrovni databáze. Dále jsme vytvořili knihovnu liwec
pro jazyk Scala, která umožňuje vytvářet grafická uživatelská rozhraní čistě v
tomto jazyce bez zbytečné komplexity.

\chapter*{Zadání}
\addcontentsline{toc}{chapter}{Zadání}

Cílem ročníkové práce bude vytvoření informačního systému pro školní jídelny (případně další hromadná stravovací zařízení).
Pracovníkům jídelny bude poskytovat možnost správy seznamu strávníků a jídel.
Strávníkům bude umožňovat volbu jídla na konkrétní dny.
Dle konzultací s potenciálními uživateli a dle analýzy existujících produktů bude zvolena další funkcionalita.
Součástí ročníkové práce bude rovněž hardwarové řešení pro výdej jídel pomocí identifikace fyzickými tokeny.
