\chapter{Instalace}

Mocasys je komplexní projekt složený z několika částí, tudíž je instalace
komplikovaný proces zahrnující vícero programovacích jazyků a prostředí.
Momentálně je jediným podporovaným operačním systémem pro vývoj Mocasysu a
spuštění serverových částí GNU/Linux.

Tyto instrukce byly ověřeny na Arch Linuxu a balíčcích z května 2019. Návod
předpokládá, že máte stažené všechny repozitáře projektu ve struktuře složek
určené submoduly.

\section{Instalace Backendu}

\begin{itemize}
    \item Nejprve nainstalujte PostgreSQL \citep[viz][]{PostgreSQL} verze 11 a
    vytvořte prázdné databázové schéma.
    
    \item Stáhněte, zkompilujte a nainstalujte rozšíření PostgreSQL
    temporal\_tables \citep[viz][]{TemporalTables}.

    \item Ve složce \texttt{mocasys-backend} spusťte skript
    \texttt{setup\_db.sh}. Skriptu můžete dát další parametry, které budou
    předány použitému příkazu \texttt{psql}.
\end{itemize}

\section{Instalance Middleendu}

\begin{itemize}
    \item Nainstalujte Node.js \citep[viz][]{NodeJs} a NPM \citep[viz][]{Npm}.

    \item Upravte soubor \texttt{mocasys-middleend/config/default.json} tak,
    aby obsahoval správné nastavení pro připojení k databázi. Middleend využívá
    dvě databázová spojení, \uv{middleend} a \uv{qdb}. \uv{Middleend} by měl
    být připojen pomocí uživatele, který má přístup k tabulkám
    \texttt{user\_passwords\_data} a \texttt{user\_mifare\_cards\_current}.
    \uv{Qdb} používá uživatele \texttt{uptest}, který byl vytvořen skriptem
    \texttt{setup\_db.sh}.

    \item Ve složce \texttt{mocasys-middleend} spusťte příkaz
    \verb|npm install|, což nainstaluje NPM balíčky, a
    \texttt{npm run start}, což spustí server middleendu.
\end{itemize}

\section{Instalace Frontendu}

\begin{itemize}
    \item Nainstalujte SBT \citep[viz][]{Sbt}.

    \item Ve složce \texttt{mocasys-frontend} spusťte příkaz \texttt{sbt
    devel}, poté \texttt{devserver.py}, což spustí webserver hostující
    aplikaci.

    \item Navštivte ve webovém prohlížeči stránku \texttt{http://127.0.0.1:8080}.
\end{itemize}
