\chapter{Frontend} \label{frontend}


Webové rozhraní Mocasysu je napsané jako SPA (z angl. Single Page Application)
v jazyce Scala s využitím kompilátoru Scala.js. Byl vybudován na vlastním
frameworku \uv{liwec}, který je sám postaven na JavaScript knihovně domvm
\citep[viz][]{DomVm}. Využívá přístup \uv{virtual DOM} známý například z
knihoven React a Vue.

\section{Liwec}

Liwec dělí aplikaci na tzv. komponenty, někdy nazývané widgety či zobrazení,
které reprezentují jednotlivé soběstačné části grafického rozhraní. Komponenty
jsou třídy dědící abstraktní třídu \texttt{Component}. Při změně jakéhokoliv
atributu komponenty je spuštěna funcke \texttt{render()}, její výsledek
porovnán s aktuálním stavem a změny vykresleny. Hlídání změn využívá
\texttt{Proxy} \citep[viz][][MdnJsProxy], funkcionalitu JavaScriptu, která
umožňuje obalit objekt a při jakémkoliv přísupu k objektu zavolat handler
funkci.

Liwec umožňuje programovat aplikace výhradně ve Scale. Obsahuje DSL (z angl.
Domain Specific Language) pro tvorbu HTML a CSS, které využívají flexibilitu
Scaly, což dovoluje mít pro jednu komponentu pouze jeden soubor bez nutnosti
vkládat další jazyky jako text.

HTML tagy jsou reprezentovány objekty, což dovoluje například vytvářet
\uv{pseudo-komponenty} -- funkce, které dle daných parametrů vrátí HTML tag.
Tyto funkce zabírají méně paměti než plné komponenty, nemohou ale udržovat
vnitřní stav pomocí hlídaných atributů, plní tak jinou roli než plné komponenty.

U CSS bývá problémem \uv{přelévání} stylů z jedné komponenty do druhé kvůli příliš
širokým selektorům či sdíleným jménům tříd. Tento problém je někdy řešen
komplikovanými vývojovými metodologiemi, liwec však implementuje \uv{scoped CSS},
koncept převzatý z Vue. Každému DOM elementu je přidán atribut, který unikatně
identifikuje komponentu, které je součástí. Ke každé části každého CSS selektoru
je pak přidána ekvivalentní podmínka.

CSS styly komponent jsou při kompilaci převedeny makrem na textovou reprezentaci
a uloženy do souborů, poté sloučeny externím skriptem. Výsledná aplikace tak
nemusí obsahovat kód na generování CSS.

% TODO: Make this a floating figure
\begin{code}
class Counter(var i: Int = 1) extends Component {
    def render() = scoped(div(
            button("Increment",
                   onClick := { _ => i += 1 }),
            div(s"Counted to \$i")
    ))
    cssScoped { import liwec.cssDsl._
        e.div (
            color := "crimson",
            fontSize := "20pt",
        )
    }
}
\end{code}

\subsection{Interní struktura}

Liwec je kompilován nástrojem SBT. Jeho tzv. projekt je rozdělen na několik dílčích projektů:

\begin{itemize}
	\item \emph{macros} obsahuje makra využívaná v DSL liwecu, konkrétně
	\texttt{scoped} pro úpravu HTML pro \uv{scoped CSS}. Dále pak
	\texttt{css} a \texttt{cssScoped}, makra pro zpracování CSS DSL.

	\item \emph{htmlCodegen} a \emph{cssCodegen} jsou programy, které ze
	stránek MDN získají data ve strukturované formě a vytvoří pak podle
	nich třídy a funkce, které pak jsou součástí DSL.

	\item \emph{liwec} obsahuje samotný kód obsažený v zkompilovaném výstupu,
	mimo jiné typy pro DSL, reprezentaci knihovny \texttt{domvm} ve Scale
	(tzv. \uv{binding}) a kód pro routování stránek dle URL.
\end{itemize}

\subsection{HTML DSL}

Vytváření HTML komponent pomocí DSL, sady prvků jazyka tvořících vnořený
\uv{jazyk}, je jednou z definujících vlastností liwecu. Tento přístup je
inspirován mimo jiné JavaScript knihovnou Mithril \citep[viz][]{MithrilJs} a
Scala knihovnou ScalaTags \citep[viz][]{ScalaTags}.

DSL je implementováno pomocí různých funkcí Scaly, několik z nich exklusivních
tomuto jazyku. Tagy jsou reprezentovány statickými objekty (konkrétně
\texttt{lazy val}), které lze volat jako funcke akceptující libovolné množství
argumentů a vracející objekt \texttt{Element\-VNode} z knihovny \texttt{domvm}.
Tyto objekty lze chápat jako \uv{typesafe} wrappery funkce \texttt{el} z
\texttt{domvm}.

Argumenty tagů jsou instance \uv{vlastnosti} (angl. trait)
\texttt{VNode\-Applicable[T]}, kde \texttt{T} je typ elementů, na které lze tuto
instanci použít. Každá instance \texttt{VNode\-Applicable} má metodu
\texttt{applyTo(vn: T): Unit}, která zmutuje předaný objekt. Instancí
\texttt{VNode\-Applicable} jsou mimo jiné objekty reprezentující atributy tagů a
jejich hodnoty a komponenty. Pomocí tzv. implicitních tříd lze pak jako
\texttt{VNode\-Applicable} brát i například řetězce, další elementy a vybrané
monády (\texttt{Seq} a \texttt{Option}), pokud obsahují další
\texttt{VNode\-Applicable}.

Atributy tagů jsou objekty, které po použití operátoru \texttt{:=} vrátí objekt
reprezentující název atributu i jeho hodnotu. Tento objekt je instancí
\texttt{VNode\-Applicable}, takže je možné jej použít jako argument tagů.

\subsection{CSS DSL}

Popis CSS pravidel v liwecu využívá objektů a vlastních operátorů k vytvoření
AST (z angl. Abstract Syntax Tree, syntaktický strom). CSS selektory jsou
reprezentovány instancemi \texttt{Selector}, které jsou vytvářeny primárně
pomocí globálních \uv{dynamických objektů}. Díky funkci Scaly je
\texttt{c.className} přetransformováno na \texttt{c.selectDynamic("className")},
což je v liwecu funkce vracející selektor pro CSS třídu nazvanou
\texttt{className}. Komplexní selektory lze pak vytvářet pomocí vlastních
operátorů vlastnosti \texttt{Selector}.

Selektor je možné buď zavolat jako funkci přímo, nebo zavolat jeho metodu
\texttt{->}. Obě funkce berou libovolné množství parametrů: samotné CSS
vlastnosti a další pravidla. Takto vzniklé pravidlo je pak zpracováno makry
\texttt{css} a \texttt{cssScoped}.

\section{Tabulky a formuláře}

Velkou část Frontendu tvoří rutinní tabulky a formuláře, které skoro přesně
odpovídají SQL tabulkám. Proto byly vytvořeny obecné třídy a funkce pro
vytváření těchto konstruktů.

Framework tabulek se skládá ze dvou vrstev. Nižší umožňuje z jakéhokoliv seznamu
a specifických objektů sloupců vytvořit tabulku, druhá se pak zaměřuje na
vytváření tabulek z libovolného SQL dotazu.

Framework formulářů je navržen vesměs dynamicky. Jádrem každého formuláře je
\texttt{Map[String, Any]}, slovník z řetězce na libovolný datový typ. Tento
přístup sice znatelně ztěžuje ověřování správnosti kódu při kompilaci, výsledný
kód je ale velmi krátký a framework nevyžaduje k funkčnosti žádná komplexní
makra.

\section{Hlavní menu}

Menu můžeme reprezentovat rekurzivně dvojící tříd \verb|MenuItem|
a \verb|SubMenu|. Obě tyto třídy dědí z traitu (interface v Javě) \verb|MenuNode|,
což zjednodušuje psaní algoritmů, jenž využívají instance
těchto tříd. Limitujeme tak počet případů.
\verb|MenuItem| uchovává text odkazu a funkci, která se zavolá
při kliknutí na odkaz.
\verb|SubMenu| si uchovává \verb|MenuItem|
seznam \verb|MenuNode|, abychom mohli menu reprezentovat rekurzivně.

\begin{code}
trait MenuNode
class MenuItem(value: String, action: Event => Unit) extends MenuNode
class SubMenu(item: MenuItem, children: Seq[MenuNode]) extends MenuNode
\end{code}

Celé menu můžeme předávat jako instanci \verb|SubMenu|.
Menu pak vyrenderujeme procházením do hloubky přibližně takto:

\begin{code}
def renderMenu(node: SubMenu): liwec.htmlDsl.VNodeFrag =
    for (child <- node.children) yield child match {
        case menu: SubMenu => 
            li(h4(menu.item.value, onClick := menu.item.action),
                ul(renderMenu(menu)))
        case item: MenuItem => 
            li(item.value, onClick := item.action)
    }
\end{code}

V implementaci je kořen menu renderován jinak kvůli stylování,
princip ale zůstává stejný. Některé html atributy byly v ukázce kódu vynechány.

\section{Globální stav}

Stav aplikace udržuje instance třídy \verb|AppStateCls|, ke které se přistupuje skrz
proměnnou \verb|AppState|. Zde jsou uloženy údaje jako uživatel, session token z
Middleendu a instance \verb|ApiClient|. Dále poskytuje abstrakční vrstvu nad tímto api klientem,
abychom mohli zachytávat chyby a zobrazovat je globálně jako \hyperref[userMessages]{zprávy uživateli}.

\section{Zprávy uživateli} \label{userMessages}

Zprávy uživateli (např. chybové) jsou renderované komponentou \verb|Messenger|, ke které lze
přistupovat z \verb|AppState|. Zprávy jsou předávány jako instance traitu \verb|Message|,
jenž po konkrétních třídách požaduje implementovat \verb|render| a \verb|duration|.

Volitelná metoda traitu \verb|Message| \verb|setCancel| umožňuje obdržet funkci k ručnímu odstranění.
Toho je například využíváno u zprávy \verb|OfflineMessage|, která se zobrazí,
pokud uživatelovo zařízení ztratí připojení. Tj., když zavolá se \verb|offline| event.
\verb|OfflineMessage| pak implementuje \verb|online| event, v jehož handleru zavolá předanou
odstraňovací funkci.

Skrz tento třídový návrh můžeme docílit různého stylování s např ikonami a dodatečnými akcemi.
\verb|Messenger| si zprávy ukládá do slovníku \verb|Map[Int, Message]|, kde klíč je náhodně
generovaný. Společně s přidáním do slovníku je vytvořen časovač pomocí funkce \verb|setTimeout|,
která po uplynutém intervalu spustí předanou funkci k odstranění ze slovníku zpráv.
Návratová hodnota funkce \verb|setTimeout| je handler, který když předáme funkci \verb|clearTimeout|,
tak zrušíme odpočet.
Datový typ \verb|duration| je \verb|Option[Int]| a pokud je hodnota této proměnné \verb|None|,
zpráva nebude automaticky odstraněna.

\section{Drag'n'drop operace}

Některé stránky (přiřazování jídel, úprava permisí) využívají operaci drag'n'\-drop (česky táhni a pusť)
k usnadnění a zpříjemnění uživatelského rozhraní. Táhnutý prvek musí implementovat událost
\verb|ondragstart| a musí mít atribut \verb|draggable|. Místa, kam může být táhnutý prvek umístěn,
pak musí implementovat události \verb|ondrop| a \verb|ondragover| \citep[viz][]{W3DnD}.
Dále se hodí implementovat na \verb|ondragleave| na místě puštení, abychom mohli například
vrátit bravu okraje na původní barvu, který jsme obarvili při \verb|ondragover|.

\section{Umělecký styl}

\begin{figure} \centering
    \def\svgwidth{\columnwidth}
    \input{mocasys_logo.pdf.tex}
\caption{Logo Mocasys}
\label{logoMocasys}
\end{figure}

\begin{figure} \centering
    \includegraphics[width=145mm]{../img/maliwan}
\caption{Paleta Frontendu}
\label{colorScheme}
\end{figure}

Paleta barev byla inspirována výrobcem zbraní Maliwan v počítačové hře Borderlands \citep[viz][]{Maliwan}.
Konečná paleta byla získána nástrojem s adresou \href{https://colormind.io}{colormind.io}, což je generátor palet
využívající strojové učení. Finální paleta je vidět na obrázku~\ref{colorScheme}.
